\section{Atividades previstas}

A tabela abaixo resume as etapas de desenvolvimento previstas no plano de trabalho proposto para este projeto.
Cada ``X'' no cronograma corresponde a 1 m�s de execu��o. \\

\begin{center}
  \begin{footnotesize}
  \renewcommand{\arraystretch}{1.4}
  \begin{tabular}{|c|l||c|c|c|c|c|c|c|c|c|c|c|c|}
    \hline
      {} & {\bf Atividade} &
      {\bf 01} & {\bf 02} & {\bf 03} & {\bf 04} & {\bf 05} & {\bf 06} &
      {\bf 07} & {\bf 08} & {\bf 09} & {\bf 10} & {\bf 11} & {\bf 12} \\
    \hline
    \hline
    1 & Projeto mec�nico b�sico & X & & & & & & & & & & & \\
    \hline
    2 & Constru��o do prot�tipo & & X & & & & & & & & & & \\
    \hline
    3 & An�lise de algoritmos   & X & X & X & & & & & & & & & \\
    \hline
    4 & Integra��o do sistema   & & & X & X & & & & & & & & \\
    \hline
    5 & Projeto da eletr�nica embarcada & X & X & X & & & & & & & & & \\
    \hline
    6 & Integra��o do m�dulo de sensores & & & X & X & X & X & & & & & & \\
    \hline
    7 & Constru��o dos m�dulos de sensoriamento & & & X & X & X & X & & & & & & \\
    \hline
    8 & Integra��o dos m�dulos e testes & & & & & & X & & & & & & \\
    \hline
  \end{tabular}
  \label{etapas}
  \end{footnotesize}

  Tabela de etapas e cronograma de execu��o.
  \renewcommand{\arraystretch}{1}
\end{center}

Como indicador f�sico do cumprimento da 1a. etapa, previsto para ser executada durante o 1o. semestre, estava
prevista a disponibiliza��o do seguinte \emph{entreg�vel}: \\

\block{{\bf Entreg�vel \#1}}{
  Relat�rio t�cnico detalhando o projeto b�sico e o projeto executivo. O projeto executivo inclui os arquivos
  fontes no formato \textsc{SolidWorks} necess�rios para a fabrica��o do sistema.
}

\bigskip%
Este \emph{entreg�vel} foi disponibilizado no reposit�rio do projeto em 02/jan/2014. \\
