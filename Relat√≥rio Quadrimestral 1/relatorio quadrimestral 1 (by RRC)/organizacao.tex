%\section{Organiza��o}

A equipe respons�vel pelo desenvolvimento deste projeto est� organizada em 5 grupos:
%(vide organograma do projeto em anexo):

\begin{enumerate}
  \item Grupo de mec�nica. Respons�vel pelo projeto e especifica��es dos componentes mec�nicos do sistema
      rob�tico. Este grupo � composto por
      \begin{itemize}
        \item 1 coordenador (professor doutor);
        \item 1 engenheiro (aluno de mestrado) e
        \item 1 empresa de consultoria (Rodrigo).
      \end{itemize}

  \item Grupo de eletr�nica. Respons�vel pelo projeto e especifica��es dos componentes eletr�nicos do
      sistema rob�tico. Este grupo � composto por
      \begin{itemize}
        \item 1 coordenador (professor doutor);
        \item 1 engenheiro (aluno de mestrado) e
        \item 1 aluno (mestrado).
      \end{itemize}

  \item Grupo de pot�ncia. Respons�vel pelo projeto e especifica��es do sistema de alimenta��o do sistema
      rob�tico. Este grupo � composto por
      \begin{itemize}
        \item 1 coordenador (professor doutor) e
        \item 1 engenheiro (aluno de mestrado).
      \end{itemize}

  \item Grupo de software. Respons�vel pelo projeto e codifica��o da interface gr�fica do sistema rob�tico.
      Este grupo � composto por
      \begin{itemize}
        \item 1 coordenador (professor doutor);
        \item 1 engenheiro (aluno de mestrado) e
        \item 3 alunos (gradua��o e mestrado).
      \end{itemize}

  \item Grupo de processamento de sinais. Respons�vel pelo desenvolvimento dos algoritmos necess�rios para
      o tratamento dos sinais aquisitados pelos sensores do sistema rob�tico. Este grupo � composto por
      \begin{itemize}
        \item 2 coordenadores (professores doutores);
        \item 1 engenheiro (p�s-doutorado) e
        \item 5 alunos (mestrado, doutorado e p�s-doutorado).
      \end{itemize}
\end{enumerate}
